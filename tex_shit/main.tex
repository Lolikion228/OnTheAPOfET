\documentclass{article}
\usepackage[utf8]{inputenc}
\usepackage[russian]{babel}
\usepackage{booktabs}
\usepackage{amsmath}
\usepackage{graphicx}
\usepackage{float}
\usepackage{ntheorem}



\theoremstyle{break}
\theorembodyfont{\normalfont}
\theoremheaderfont{\bfseries}
\newtheorem{example}{Пример}


\usepackage[
left=1.5cm,    % отступ слева
right=1.5cm,   % отступ справа
top=2cm,     % отступ сверху
bottom=2cm   % отступ снизу
]{geometry}


\begin{document}

Всюду далее $g(x) = \ln(1+x^2)$.

\begin{example}
Пусть $F_1(x) = F(x)$ и $F_{2,n}(x) = F(x(1 + \frac{h_2}{\sqrt{n}})+ \frac{h_1}{\sqrt{n}} )$, где $F(x)$ - ф-я стандартного нормального распределения.\\
Пусть $\alpha = 0.05$.
С помощью численного интегрирования были получены значения:
\begin{equation*}
	J_1 = 0.810113 \hspace{0.1in} J_2 = 1.154885 \hspace{0.1in}  J_3=0.763368 \hspace{0.1in}
	\frac{J^{*}_1(h_1)}{{h_1}^2} = 0.227179   \hspace{0.1in} \frac{J^{*}_2(h_2)}{{h_2}^2} = 0.147563
\end{equation*}
В таблицах 1 и 2 представлены значения асимптотической мощности, вычисленные по формуле,
и эмпирические мощности, полученные в результате $N = 5000$ повторений статистического
моделирования двух выборок размера $n = 100$, $400$, $900$, $1600$, $2000$, $3000$. Критическое значение
критерия $nT_n$ вычислялось с помощью $M=5000$ случайных перестановок, как $1-\alpha$ квантиль эмпирического распределения $nT_n$, построенного по выборке размера $M$. Для вычисления значений в таблице 1 было взято значение $h_1=0$, для вычисления значений в таблице 2 было взято значение $h_2=0$.
\end{example}


\begin{table}[!h]
	\caption{Значения эмпирической (EP) и асимптотической (AP) мощности ($h_1=0$)}
	\centering
	\hspace*{-1.0cm}\begin{tabular}
		{lrrrrrrrrrr}
		\toprule
		& h2=0.5 & h2=1.0 & h2=1.5 & h2=2.0 & h2=2.5 & h2=3.0 & h2=3.5 & h2=4.0 & h2=4.5 & h2=5.0\\
		\midrule
		EP (n=100) & 0.055 & 0.060 & 0.085 & 0.105 & 0.155 & 0.215 & 0.275 & 0.359 & 0.445 & 0.545\\
		EP (n=400) & 0.057 & 0.069 & 0.096 & 0.138 & 0.209 & 0.299 & 0.415 & 0.534 & 0.657 & 0.757\\
		EP (n=900) & 0.049 & 0.067 & 0.095 & 0.143 & 0.212 & 0.316 & 0.442 & 0.579 & 0.706 & 0.810\\
		EP (n=1600) & 0.045 & 0.059 & 0.074 & 0.126 & 0.181 & 0.297 & 0.408 & 0.554 & 0.689 & 0.796\\
		EP (n=2000) & 0.051 & 0.066 & 0.103 & 0.143 & 0.220 & 0.325 & 0.467 & 0.603 & 0.741 & 0.837\\
		EP (n=3000) & 0.056 & 0.067 & 0.102 & 0.157 & 0.235 & 0.337 & 0.489 & 0.637 & 0.760 & 0.868\\
		\addlinespace
		AP & 0.058 & 0.082 & 0.124 & 0.183 & 0.260 & 0.351 & 0.453 & 0.557 & 0.658 & 0.749\\
		\bottomrule
	\end{tabular}
\end{table}







\begin{table}[!h]
	\centering
	\caption{Значения эмпирической (EP) и асимптотической (AP) мощности ($h_2=0$)}
	\hspace*{-1.0cm}\begin{tabular}
		{lrrrrrrrrrr}
		\toprule
		& h1=1.0 & h1=2.0 & h1=3.0 & h1=4.0 & h1=5.0 & h1=6.0 & h1=7.0 & h1=8.0 & h1=9.0 & h1=10.0\\
		\midrule
		EP (n=100) & 0.083 & 0.236 & 0.476 & 0.729 & 0.904 & 0.974 & 0.997 & 1.000 & 1 & 1\\
		EP (n=400) & 0.115 & 0.275 & 0.558 & 0.783 & 0.932 & 0.985 & 1.000 & 0.999 & 1 & 1\\
		EP (n=900) & 0.106 & 0.291 & 0.531 & 0.782 & 0.922 & 0.985 & 0.997 & 1.000 & 1 & 1\\
		EP (n=1600) & 0.102 & 0.271 & 0.524 & 0.759 & 0.914 & 0.982 & 0.996 & 1.000 & 1 & 1\\
		EP (n=2000) & 0.116 & 0.278 & 0.531 & 0.781 & 0.926 & 0.985 & 0.998 & 1.000 & 1 & 1\\
		EP (n=3000) & 0.102 & 0.270 & 0.541 & 0.762 & 0.915 & 0.983 & 0.997 & 1.000 & 1 & 1\\
		\addlinespace
		AP & 0.100 & 0.257 & 0.499 & 0.742 & 0.904 & 0.975 & 0.995 & 0.999 & 1 & 1\\
		\bottomrule
	\end{tabular}
\end{table}


\newpage

\begin{example}
	Пусть $F_1(x) = F(x)$ и $F_{2,n}(x) = F(x(1 + \frac{h_2}{\sqrt{n}})+ \frac{h_1}{\sqrt{n}} )$, где $F(x)$ - ф-я стандартного распределения Коши.\\
	Пусть $h_1=0$, $\alpha = 0.05$.
	С помощью численного интегрирования были получены значения:
	\begin{equation*}
		J_1 = 2.197224 \hspace{0.1in} J_2 = 9.577512 \hspace{0.1in}  J_3=6.881056 \hspace{0.1in}
		\frac{J^{*}_1(h_1)}{{h_1}^2} = 0.111110   \hspace{0.1in} \frac{J^{*}_2(h_2)}{{h_2}^2} = 0.111105
	\end{equation*}
	В таблицах 3 и 4 представлены значения асимптотической мощности, вычисленные по формуле,
	и эмпирические мощности, полученные в результате $N = 5000$ повторений статистического
	моделирования двух выборок размера $n = 100$, $400$, $900$, $1600$, $2000$, $3000$. Критическое значение
	критерия $nT_n$ вычислялось с помощью $M=5000$ случайных перестановок, как $1-\alpha$ квантиль эмпирического распределения $nT_n$, построенного по выборке размера $M$. Для вычисления значений в таблице 3 было взято значение $h_1=0$, для вычисления значений в таблице 4 было взято значение $h_2=0$.
\end{example}

\begin{table}[!h]
	\caption{Значения эмпирической (EP) и асимптотической (AP) мощности ($h_1=0$)}
	\centering
	\hspace*{-1.0cm}\begin{tabular}
		{lrrrrrrrrrr}
		\toprule
		& h2=1.0 & h2=2.0 & h2=3.0 & h2=4.0 & h2=5.0 & h2=6.0 & h2=7.0 & h2=8.0 & h2=9.0 & h2=10.0\\
		\midrule
		EP (n=100) & 0.058 & 0.080 & 0.127 & 0.189 & 0.256 & 0.327 & 0.412 & 0.498 & 0.574 & 0.645\\
		EP (n=400) & 0.048 & 0.083 & 0.137 & 0.218 & 0.332 & 0.452 & 0.568 & 0.675 & 0.773 & 0.843\\
		EP (n=900) & 0.056 & 0.097 & 0.165 & 0.265 & 0.403 & 0.536 & 0.665 & 0.783 & 0.865 & 0.924\\
		EP (n=1600) & 0.066 & 0.107 & 0.179 & 0.285 & 0.442 & 0.585 & 0.716 & 0.818 & 0.907 & 0.949\\
		EP (n=2000) & 0.056 & 0.109 & 0.170 & 0.298 & 0.432 & 0.571 & 0.719 & 0.837 & 0.907 & 0.951\\
		EP (n=3000) & 0.057 & 0.108 & 0.168 & 0.297 & 0.445 & 0.597 & 0.736 & 0.856 & 0.924 & 0.965\\
		\addlinespace
		AP & 0.066 & 0.115 & 0.201 & 0.319 & 0.461 & 0.608 & 0.741 & 0.846 & 0.918 & 0.961\\
		\bottomrule
	\end{tabular}
\end{table}



\begin{table}[!h]
	\centering
	\caption{Значения эмпирической (EP) и асимптотической (AP) мощности ($h_2=0$)}
	\hspace*{-1.0cm}\begin{tabular}
		{lrrrrrrrrrr}
		\toprule
		& h1=1.0 & h1=2.0 & h1=3.0 & h1=4.0 & h1=5.0 & h1=6.0 & h1=7.0 & h1=8.0 & h1=9.0 & h1=10.0\\
		\midrule
		EP (n=100) & 0.072 & 0.113 & 0.208 & 0.339 & 0.516 & 0.671 & 0.810 & 0.907 & 0.965 & 0.988\\
		EP (n=400) & 0.048 & 0.088 & 0.169 & 0.290 & 0.456 & 0.640 & 0.772 & 0.891 & 0.954 & 0.982\\
		EP (n=900) & 0.064 & 0.115 & 0.188 & 0.321 & 0.499 & 0.680 & 0.818 & 0.912 & 0.967 & 0.990\\
		EP (n=1600) & 0.061 & 0.112 & 0.199 & 0.342 & 0.516 & 0.668 & 0.815 & 0.907 & 0.966 & 0.991\\
		EP (n=2000) & 0.055 & 0.111 & 0.197 & 0.334 & 0.499 & 0.665 & 0.812 & 0.914 & 0.964 & 0.990\\
		EP (n=3000) & 0.074 & 0.115 & 0.218 & 0.360 & 0.520 & 0.699 & 0.829 & 0.926 & 0.972 & 0.992\\
		\addlinespace
		AP & 0.066 & 0.116 & 0.201 & 0.319 & 0.461 & 0.608 & 0.741 & 0.846 & 0.918 & 0.961\\
		\bottomrule
	\end{tabular}
\end{table}


\newpage
sfa

\begin{table}[!h]
	\centering
	\caption{Energy Test for Cauchy Distribution ( fixed h1=0 )}
	\centering
	\hspace*{-1.0cm}\begin{tabular}
		[t]{lrrrrrrrrrr}
		\toprule
		& h2=1.0 & h2=2.0 & h2=3.0 & h2=4.0 & h2=5.0 & h2=6.0 & h2=7.0 & h2=8.0 & h2=9.0 & h2=10.0\\
		\midrule
		EP (n=100) & 0.051 & 0.080 & 0.116 & 0.177 & 0.250 & 0.322 & 0.399 & 0.479 & 0.553 & 0.643\\
		EP (n=400) & 0.052 & 0.085 & 0.148 & 0.240 & 0.336 & 0.459 & 0.588 & 0.705 & 0.783 & 0.864\\
		EP (n=900) & 0.057 & 0.095 & 0.160 & 0.271 & 0.385 & 0.537 & 0.658 & 0.780 & 0.868 & 0.930\\
		EP (n=1600) & 0.071 & 0.116 & 0.197 & 0.316 & 0.450 & 0.598 & 0.735 & 0.842 & 0.913 & 0.953\\
		EP (n=2000) & 0.068 & 0.100 & 0.175 & 0.301 & 0.443 & 0.593 & 0.723 & 0.841 & 0.907 & 0.959\\
		
		EP (n=3000) & 0.060 & 0.103 & 0.180 & 0.313 & 0.452 & 0.609 & 0.740 & 0.855 & 0.925 & 0.966\\
		\addlinespace
		AP & 0.066 & 0.115 & 0.201 & 0.319 & 0.461 & 0.608 & 0.741 & 0.846 & 0.918 & 0.961\\
		\bottomrule
	\end{tabular}
\end{table}


\begin{table}[!h]
	\centering
	\caption{Energy Test for Cauchy Distribution ( fixed h2=2 )}
	\centering
	\hspace*{-1.0cm}\begin{tabular}
		[t]{lrrrrrrrrrr}
		\toprule
		& h1=1.0 & h1=2.0 & h1=3.0 & h1=4.0 & h1=5.0 & h1=6.0 & h1=7.0 & h1=8.0 & h1=9.0 & h1=10.0\\
		\midrule
		EP (n=100) & 0.093 & 0.114 & 0.188 & 0.290 & 0.413 & 0.568 & 0.708 & 0.828 & 0.913 & 0.954\\
		EP (n=400) & 0.127 & 0.167 & 0.269 & 0.367 & 0.522 & 0.677 & 0.803 & 0.893 & 0.957 & 0.987\\
		EP (n=900) & 0.114 & 0.168 & 0.262 & 0.385 & 0.539 & 0.692 & 0.823 & 0.909 & 0.962 & 0.987\\
		EP (n=1600) & 0.128 & 0.187 & 0.284 & 0.416 & 0.565 & 0.711 & 0.840 & 0.919 & 0.964 & 0.991\\
		EP (n=2000) & 0.125 & 0.186 & 0.270 & 0.401 & 0.541 & 0.702 & 0.843 & 0.924 & 0.968 & 0.989\\
		
		EP (n=3000) & 0.121 & 0.175 & 0.270 & 0.392 & 0.556 & 0.703 & 0.832 & 0.927 & 0.966 & 0.985\\
		\addlinespace
		AP & 0.132 & 0.183 & 0.269 & 0.384 & 0.518 & 0.653 & 0.773 & 0.866 & 0.929 & 0.967\\
		\bottomrule
	\end{tabular}
\end{table}



\end{document}